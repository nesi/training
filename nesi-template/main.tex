\documentclass[aspectratio=169]{beamer}
%\documentclass[aspectratio=169,notes=only]{beamer}
\usepackage[english]{layout}
\usepackage[utf8]{inputenc}
\usepackage[english]{babel}
\usepackage[T1]{fontenc}
\usepackage{amsmath, soul, color, multicol, type1cm, verbatim, latexsym, dsfont, float, listings,alltt,tikz, lmodern, textcomp}
\usepackage[official]{eurosym}
\usepackage[export]{adjustbox}
\usepackage[caption = false]{subfig}
%\usepackage{beamerthemesplit}
%\usetheme{Frankfurt}
\usecolortheme{lily}
%\usefonttheme{structuresmallcapsserif}
\usefonttheme{professionalfonts}
\setbeamercovered{transparent}

%NeSI Colors <-------------------------------------------------------------------------------------
\usecolortheme{lily}
\usecolortheme[RGB={47, 68, 71}]{structure} 
\definecolor{nesidark}{HTML}{2F4447}
\definecolor{nesilight}{HTML}{CED9DF}
\definecolor{nesigrey}{gray}{0.7}
\definecolor{nesilightgrey}{gray}{0.98}
\definecolor{nesidarkgrey}{gray}{0.3}
\definecolor{nesiblue}{HTML}{2B9FC2}
\setbeamercolor{block title}{fg=black,bg=nesigrey}
\setbeamercolor{block body}{bg=nesilightgrey,fg=nesidarkgrey}
\setbeamercolor{block body alerted}{bg=white,fg=black}
\setbeamercolor{alerted text}{bg=white,fg=black}
\addtobeamertemplate{frametitle}{\vskip+1.2ex}{}
%NeSI Custom Code Hightlight <---------------------------------------------------------------------------------------
\lstdefinestyle{customcode}{
  belowcaptionskip=1\baselineskip,
  breaklines=true,
  xleftmargin=\parindent,
  showstringspaces=false,
  basicstyle=\ttfamily,
  keywordstyle=\bfseries\color{green!40!black},
  commentstyle=\itshape\color{purple!40!black},
  identifierstyle=\color{blue},
  stringstyle=\color{orange},
}
%NeSI Title <---------------------------------------------------------------------------------------
\setbeamerfont{title}{size=\huge}
\frenchspacing
\hyphenation{NeSI}
%NeSI Template parameters <-------------------------------------------------------------------------
\setbeamertemplate{blocks}[default]
\useinnertheme{circles}
\setbeamertemplate{title page}[default][center,rounded=false,shadow=false]
\newcommand\BackgroundPicture[1]{%
\setbeamertemplate{background}{%
\parbox[c][\paperheight]{\paperwidth}{%
\vfill \hfill \includegraphics[height=\paperheight]{#1}
\hfill \vfill
}}}

%Content Starts Here <-------------------------------------------------------------------------------
\title{NeSI Template}
%\subtitle{Computational Science team}
\author{Cool Staff Members use \LaTeX{} \\(cool.staff@nesi.org.nz)}
\date{}

\begin{document}
\BackgroundPicture{NeSI/title.png}
\begin{frame}[plain]
  \vspace{+4cm}
  \titlepage
\end{frame}

% This will generate the outline. If you have several topics, uncomment the multicols 
\BackgroundPicture{NeSI/blank-01.png}
\begin{frame}
  \frametitle{Outline}
  %\begin{multicols}{2} 
  \tableofcontents
  %\end{multicols}
\end{frame}

%%%%%%%%%%%%%%%%%%%%%%%%%%%%%%%%%%%%%%%%%%%%%%%%%%%%%%%%%%%%%%%%%%%%%%%%%%%%%%%%%%%%%%%%%%%%%%%
%%%%%%%%%%%%%%%%%%%%%%%%%%%%%%%%%%%%%%%% Some Examples %%%%%%%%%%%%%%%%%%%%%%%%%%%%%%%%%%%%%%%%
%%%%%%%%%%%%%%%%%%%%%%%%%%%%%%%%%%%%%%%%%%%%%%%%%%%%%%%%%%%%%%%%%%%%%%%%%%%%%%%%%%%%%%%%%%%%%%%
\section{Introduction}
\subsection{About Something}
% Include Graphics <---
\begin{frame}[t]
  \frametitle{About Something}
  \begin{center}
    \includegraphics[width=0.9\textwidth]{NeSI/nesi_logo.png}
  \end{center}
\end{frame}

% List with bullets (itemize) <---
% The [<+-| alert@+>] will create a new slide for each element with highlighting 
\begin{frame}[t]
  \frametitle{About Something}
    \begin{block}{About Something}
      \begin{itemize}%[<+-| alert@+>]
	  \item foo
	  \item bar
	  \item baz
	  \item qux
    \end{itemize}
  \end{block}
\end{frame}

% List with numbers (enumerate) <---
% The [<+-| alert@+>] will create a new slide for each element with highlighting 
\begin{frame}[t]
  \frametitle{About Something}
    \begin{block}{About Something}
      \begin{enumerate}%[<+-| alert@+>]
	  \item inky
	  \item pinky
	  \item blinky
	  \item clyde
    \end{enumerate}
  \end{block}
\end{frame}

% If you want to include some code (verbatim or listings) you will need to add "fragile" in the frame options <---
\section{Examples}
\subsection{verbatim example}
\begin{frame}[fragile]
  \frametitle{Verbatim example}
    \begin{block}{Job Description Example : SMP}
      \begin{small}
        \begin{verbatim}
#!/bin/bash
#SBATCH -o /share/src/app/example/app-%j.%N.out
#SBATCH -D /share/src/app/example
#SBATCH -J MIGRATESMP
#SBATCH --get-user-env
#SBATCH --ntasks=1
#SBATCH --cpu_bind=core
#SBATCH --export=NONE
#SBATCH --time=08:00:00
ml load migrate/3.4.4-smp
export OMP_NUM_THREADS=4
srun migrate-n-thread parmfile.testml -nomenu
      \end{verbatim}
    \end{small}
  \end{block}
\end{frame}

\subsection{listings example}
\begin{frame}[fragile]
  \frametitle{Listings example}
  \begin{block}{Generate Trace File with ITAC in the Cluster - Non-instrumented}
    \begin{footnotesize}
      \lstset{language=bash,escapechar=@,style=customcode}
      \begin{lstlisting}
#!/bin/bash
#SBATCH -J ITAC
#SBATCH -A nesi99999
#SBATCH --time=6:00:00
#SBATCH --mem-per-cpu=6144
#SBATCH -C ib
#SBATCH --ntasks=48
#SBATCH --nodes=2
ml intel/2015a
ml VTune/2015_update2
ml itac/9.0.3.051
source itacvars.sh impi4
unset I_MPI_PMI_LIBRARY
mpiexec.hydra -trace binary [my app options]
      \end{lstlisting}
    \end{footnotesize}
  \end{block}
\end{frame}

\subsection{Divider examples}
\BackgroundPicture{NeSI/divider-01.png}
\begin{frame}[fragile]{}
  \begin{center}
    \Huge{\textbf{Divider 1 example}}
  \end{center}
\end{frame}

\note[itemize]{
\item The beauty of LXD is that you can run anywhere you running Linux.
\item No HW acceleration requirements (VT)
\item Can run inside VMs, all architectures, etc.
}

\BackgroundPicture{NeSI/divider-02.png}
\begin{frame}[fragile]{}
  \begin{center}
    \Huge{\textbf{Divider 2 example}}
  \end{center}
\end{frame}

\BackgroundPicture{NeSI/divider-03.png}
\begin{frame}[fragile]{}
  \begin{center}
    \Huge{\textbf{Divider 3 example}}
  \end{center}
\end{frame}

\BackgroundPicture{NeSI/divider-04.png}
\begin{frame}[fragile]{}
  \begin{center}
    \Huge{\textbf{Divider 4 example}}
  \end{center}
\end{frame}

\BackgroundPicture{NeSI/sidebar-01.png}
\begin{frame}[t]{}
  \frametitle{sidebar title}
  \begin{columns}
    \begin{column}{0.25\textwidth}
  
    \end{column}
    \begin{column}{0.75\textwidth}\centering
    \vspace*{+1.5cm}
    \begin{Huge}
      \begin{itemize}%[<+-| alert@+>]
        \item a
        \item b
        \item c
      \end{itemize}
    \end{Huge}
    \end{column}
  \end{columns}
\end{frame}

\BackgroundPicture{NeSI/sidebar-02.png}
\begin{frame}[t]{}
  \frametitle{sidebar title}
  \begin{columns}
    \begin{column}{0.25\textwidth}
  
    \end{column}
    \begin{column}{0.75\textwidth}\centering
    \vspace*{+1.5cm}
    \begin{Huge}
      \begin{itemize}%[<+-| alert@+>]
        \item a
        \item b
        \item c
      \end{itemize}
    \end{Huge}
    \end{column}
  \end{columns}
\end{frame}

\BackgroundPicture{NeSI/sidebar-03.png}
\begin{frame}[t]{}
  \frametitle{sidebar title}
  \begin{columns}
    \begin{column}{0.25\textwidth}
  
    \end{column}
    \begin{column}{0.75\textwidth}\centering
    \vspace*{+1.5cm}
    \begin{Huge}
      \begin{itemize}%[<+-| alert@+>]
        \item a
        \item b
        \item c
      \end{itemize}
    \end{Huge}
    \end{column}
  \end{columns}
\end{frame}

\BackgroundPicture{NeSI/sidebar-04.png}
\begin{frame}[t]{}
  \frametitle{sidebar title}
  \begin{columns}
    \begin{column}{0.25\textwidth}
  
    \end{column}
    \begin{column}{0.75\textwidth}\centering
    \vspace*{+1.5cm}
    \begin{Huge}
      \begin{itemize}%[<+-| alert@+>]
        \item a
        \item b
        \item c
      \end{itemize}
    \end{Huge}
    \end{column}
  \end{columns}
\end{frame}

\note[itemize]{
\item a
\item b
\item c
}

% End <-----------------------------------------------------------------------------------------
\BackgroundPicture{NeSI/blank-02.png}
\begin{frame}[plain]
  \begin{center}
    \vspace*{+2cm}
    {\Huge Thank You}\\
    \vspace*{+1cm}
    \includegraphics[width=100pt]{NeSI/nesi_logo.png}
  \end{center}
\end{frame}

\end{document} 