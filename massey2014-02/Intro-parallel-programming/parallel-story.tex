\documentclass{beamer}
\usepackage[english]{layout}
\usepackage[utf8]{inputenc}
\usepackage[english]{babel}
\usepackage[T1]{fontenc}
\usepackage{amsmath, soul, color, multicol, type1cm, verbatim, latexsym, dsfont, float, listings}
\usepackage[official]{eurosym}
\usepackage{algorithmic}
\usepackage{algorithm2e}
\usepackage{float}
\usepackage{beamerthemesplit}
\usepackage{graphicx}
\usepackage[T1]{fontenc}
\usetheme{Frankfurt}
\usecolortheme{lily}
%\usefonttheme{structuresmallcapsserif}
\usefonttheme{professionalfonts}
\setbeamercovered{transparent}

%NeSI Colors <----------------------------------------------------------------------------
\usecolortheme[RGB={47, 68, 71}]{structure} 
\definecolor{nesidark}{HTML}{2F4447}
\definecolor{nesilight}{HTML}{CED9DF}
\definecolor{nesigrey}{gray}{0.7}
\definecolor{nesilightgrey}{gray}{0.98}
\definecolor{nesidarkgrey}{gray}{0.3}
\definecolor{nesiblue}{HTML}{2B9FC2}
\setbeamercolor{block title}{fg=black,bg=nesigrey}
\setbeamercolor{block body}{bg=nesilightgrey,fg=nesidarkgrey}
\setbeamercolor{block body alerted}{bg=white,fg=black}
\setbeamercolor{alerted text}{bg=white,fg=black}

\setbeameroption{show notes}

\setbeamerfont{title}{size=\huge}
\frenchspacing
\hyphenation{NeSI}

\newenvironment{topcolumns}{\begin{columns}[t]}{\end{columns}}
\newcommand\BackgroundPicture[1]{%
\setbeamertemplate{background}{%
\parbox[c][\paperheight]{\paperwidth}{%
\vfill \hfill \includegraphics[height=0.9\paperheight]{#1}
\hfill \vfill
}}}

\setbeamertemplate{blocks}[default]%[shadow=false]
\useinnertheme{circles}
\setbeamertemplate{title page}[default][center,rounded=false,shadow=false]
%\setbeamertemplate{title page}[default][center, wd=60mm, colsep=-4bp,rounded=true]

% Fancy Footer Content    <---------------------------------------------------------------
%\setbeamertemplate{footline}{
%   \unilogo
%   \dsglogo
%   \begin{beamercolorbox}[ht=4ex,leftskip=1.4cm,rightskip=.3cm]{author in head/foot}
%    \usebeamercolor{nesiblue}
%    \hrule
%    \vspace{0.1cm}
%    \insertdate \hfill \inserttitle \newline
%    \insertshortauthor \ - \insertshortinstitute \hfill \insertframenumber
%   \end{beamercolorbox}
%   \vspace*{0.1cm}
%} 
% Reference http://joerglenhard.wordpress.com/2011/08/04/beamer-customization-ii-footline-with-multiple-lines/
% http://joerglenhard.wordpress.com/tag/beamer/
% https://github.com/lenhard/ub-beamer

%%%%%%%%%%%%%%%%%%%%%%%%%%%%%%%%%%%%%%%%%%%%%%%%%%%%%%%%%%%%%%%%%%%%%%%%%%%%%%%%%%%%%%%%%%
%%%%%%%%%%%%%%%%%%%%%%%%%%%%%%%%%%%%%%%%%%%%%%%%%%%%%%%%%%%%%%%%%%%%%%%%%%%%%%%%%%%%%%%%%
\title{Introduction to Parallel Computing}
%\subtitle{According to NeSI}
\author{NeSI Computational Science Team \\support@nesi.org.nz}
\date{}

%%%%%%%%%%%%%%%%%%%%%%%%%%%%%%%%%%%%%%%%%%%%%%%%%%%%%%%%%%%%%%%%%%%%%%%%%%%%%%%%%%%%%%%%%%
%%%%%%%%%%%%%%%%%%%%%%%%%%%%%%%%%%%%%%%%%%%%%%%%%%%%%%%%%%%%%%%%%%%%%%%%%%%%%%%%%%%%%%%%%
\begin{document}

{
\setbeamertemplate{background canvas}{\includegraphics[height=0.99\paperheight]{NeSI_img/Slide00.png}} 
\begin{frame}[plain]
\vspace{1cm}
\titlepage
\end{frame}
}


%\BackgroundPicture{NeSI_img/SlideXX.png}
\begin{frame}
\frametitle{Outline}
\begin{multicols}{2}
   \tableofcontents
 \end{multicols}
 \end{frame}

%%%%%%%%%%%%%%%%%%%%%%%%%%%%%%%%%%%%%%%%%%%%%%%%%%%%%%%%%%%%%%%%%%%%%%%%%%%%%%%%%%%%%%%%%
%%%%%%%%%%%%%%%%%%%%%%%%%%%%%%%%%%%%%%%%%%%%%%%%%%%%%%%%%%%%%%%%%%%%%%%%%%%%%%%%%%%%%%%%%
\section{Story of computing}
%%%%%%%%%%%%%%%%%%%%%%%%%%%%%%%%%%%%%%%%%%%%%%%%%%%%%%%%%%%%%%%%%%%%%%%%%%%%%%%%%%%%%%%%%
%%%%%%%%%%%%%%%%%%%%%%%%%%%%%%%%%%%%%%%%%%%%%%%%%%%%%%%%%%%%%%%%%%%%%%%%%%%%%%%%%%%%%%%%%
%%%%%%%%%%%%%%%%%%%%%%%%%%%%%%%%%%%%%%%%%%%%%%%%%%%%%%%%%%%%%%%%%%%%%%%%%%%%%%%%%%%%%%%%%
\subsection{The beginning}
%%%%%%%%%%%%%%%%%%%%%%%%%%%%%%%%%%%%%%%%%%%%%%%%%%%%%%%%%%%%%%%%%%%%%%%%%%%%%%%%%%%%%%%%%

%%%%%%%%%%%%%%%%%%%%%%%%%%%%%%%%%%%%%%%%%%%%%%%%%%%%%%%%%%%%%%%%%%%%%%%%%%%%%%%%%%%%%%%%%
\frame[t]
{
	\frametitle{Story of computing}
	\framesubtitle{The Beginning}
   \begin{block}{Alan Turing}
   \begin{center}
 \includegraphics[width=250pt]{fig/turing.jpg}
   \end{center}
  \end{block}

}
%%%%%%%%%%%%%%%%%%%%%%%%%%%%%%%%%%%%%%%%%%%%%%%%%%%%%%%%%%%%%%%%%%%%%%%%%%%%%%%%%%%%%%%%%

%%%%%%%%%%%%%%%%%%%%%%%%%%%%%%%%%%%%%%%%%%%%%%%%%%%%%%%%%%%%%%%%%%%%%%%%%%%%%%%%%%%%%%%%%
\frame[t]
{
	\frametitle{Story of computing}
	\begin{block}{Turing's Machine}
   	\begin{itemize}
		\item 	During World War II, Alan Turing - a British mathematician started to work
		with Britain's code-breaking centre and deciphered Germany's U-boat Enigma, saving 
		battle of the Atlantic! 
	  	\item	He conceived the principles of modern computers and brought about
	  	his famous "Turing's Machine" in 1936. 
	  	\item "Turing's Machine" is a hypothetical device that manipulates symbols 
	  	on a strip of tape according to a table of rules which could simulate the logic of
	  	almost any computer algorithm.
   	\end{itemize}	

   	  
	\end{block}
}
%%%%%%%%%%%%%%%%%%%%%%%%%%%%%%%%%%%%%%%%%%%%%%%%%%%%%%%%%%%%%%%%%%%%%%%%%%%%%%%%%%%%%%%%%

%%%%%%%%%%%%%%%%%%%%%%%%%%%%%%%%%%%%%%%%%%%%%%%%%%%%%%%%%%%%%%%%%%%%%%%%%%%%%%%%%%%%%%%%%
\frame[t]
{
	\frametitle{Story of computing}
	\begin{block}{Z1}
   		\begin{itemize}
			\item On the other side of the spectrum Konrad Zuse - 
			a German civil engineer in Berlin - started dreaming of a machine that could do 
			mechanical calculations.
			\item He started working in his parents' apartment and built his 
			first electro-mechanical computer – Z1 in the same year 1936.
			\item It was a floating point binary mechanical calculator with limited 
			programmability, reading instructions from a perforated 35 mm film.
		\end{itemize}
		
	\end{block}
}
%%%%%%%%%%%%%%%%%%%%%%%%%%%%%%%%%%%%%%%%%%%%%%%%%%%%%%%%%%%%%%%%%%%%%%%%%%%%%%%%%%%%%%%%%

%%%%%%%%%%%%%%%%%%%%%%%%%%%%%%%%%%%%%%%%%%%%%%%%%%%%%%%%%%%%%%%%%%%%%%%%%%%%%%%%%%%%%%%%%
\frame[t]
{
	\frametitle{Story of computing}
   	\begin{block}{Konrad Zuse}
   		\begin{center}
 			\includegraphics[width=250pt]{fig/zuse.jpg}
   		\end{center}
   		So we have a rare situation of two people coming from two different 
		backgrounds and laying foundations for computer science. 
  	\end{block}
}
%%%%%%%%%%%%%%%%%%%%%%%%%%%%%%%%%%%%%%%%%%%%%%%%%%%%%%%%%%%%%%%%%%%%%%%%%%%%%%%%%%%%%%%%%

%%%%%%%%%%%%%%%%%%%%%%%%%%%%%%%%%%%%%%%%%%%%%%%%%%%%%%%%%%%%%%%%%%%%%%%%%%%%%%%%%%%%%%%%%
\frame[t]
{
  \frametitle{Story of computing}
    \begin{block}{First computers}   		
      \begin{itemize}
			\item Invention of transistors in 1947 by John Bardeen, Walter 
			Brattain \& Wiliam Shockley replaced vacuum tubes used in early computers.
			\item Computers started to reduce in size and increase in speed. 
   \end{itemize}
  \end{block}
  	\begin{center}
  		\includegraphics[width=100pt]{fig/transistor.jpg}
 	\end{center}
}
%%%%%%%%%%%%%%%%%%%%%%%%%%%%%%%%%%%%%%%%%%%%%%%%%%%%%%%%%%%%%%%%%%%%%%%%%%%%%%%%%%%%%%%%%

%%%%%%%%%%%%%%%%%

%%%%%%%%%%%%%%%%%%%%%%%%%%%%%%%%%%%%%%%%%%%%%%%%%%%%%%%%%%%%%%%%%%%%%%%%%%%%%%%%%%%%%%%%%
\subsection{Need for speed}
%%%%%%%%%%%%%%%%%%%%%%%%%%%%%%%%%%%%%%%%%%%%%%%%%%%%%%%%%%%%%%%%%%%%%%%%%%%%%%%%%%%%%%%%%

%%%%%%%%%%%%%%%%%%%%%%%%%%%%%%%%%%%%%%%%%%%%%%%%%%%%%%%%%%%%%%%%%%%%%%%%%%%%%%%%%%%%%%%%%
\frame[t]
{
  \frametitle{Story of computing}
    \framesubtitle{Need for speed}
      \begin{block}{Seymour Cray}
   \begin{itemize}
	\item There was great appetite for speed, which was fuelled by aspirations 
	of an American named Seymour Cray. 
	\item He designed the first supercomputer - Cray-1 - in 1976. He is called the 
	"father of supercomputing". 
   \end{itemize}
  \end{block}
  
    \begin{center}
  		\includegraphics[width=140pt]{fig/cray.jpg}
 	\end{center}
}
%%%%%%%%%%%%%%%%%%%%%%%%%%%%%%%%%%%%%%%%%%%%%%%%%%%%%%%%%%%%%%%%%%%%%%%%%%%%%%%%%%%%%%%%%

%%%%%%%%%%%%%%%%%%%%%%%%%%%%%%%%%%%%%%%%%%%%%%%%%%%%%%%%%%%%%%%%%%%%%%%%%%%%%%%%%%%%%%%%%
\frame[t]
{
  \frametitle{Story of computing}
      \begin{block}{Cray-1} 
        First supercomputers were monolithic in structure and architecture. 
   		\begin{center}
 			\includegraphics[width=190pt]{fig/cray1-install.jpg}
   		\end{center}  
		
  	\end{block}
  	\note {This slide is intentionally left blank.} 
}
%%%%%%%%%%%%%%%%%%%%%%%%%%%%%%%%%%%%%%%%%%%%%%%%%%%%%%%%%%%%%%%%%%%%%%%%%%%%%%%%%%%%%%%%%


%%%%%%%%%%%%%%%%%%%%%%%%%%%%%%%%%%%%%%%%%%%%%%%%%%%%%%%%%%%%%%%%%%%%%%%%%%%%%%%%%%%%%%%%%%
%%%%%%%%%%%%%%%%%%%%%%%%%%%%%%%%%%%%%%%%%%%%%%%%%%%%%%%%%%%%%%%%%%%%%%%%%%%%%%%%%%%%%%%%%%
\section{Hegelian dialectics}
%%%%%%%%%%%%%%%%%%%%%%%%%%%%%%%%%%%%%%%%%%%%%%%%%%%%%%%%%%%%%%%%%%%%%%%%%%%%%%%%%%%%%%%%%%
%%%%%%%%%%%%%%%%%%%%%%%%%%%%%%%%%%%%%%%%%%%%%%%%%%%%%%%%%%%%%%%%%%%%%%%%%%%%%%%%%%%%%%%%%%

%%%%%%%%%%%%%%%%%%%%%%%%%%%%%%%%%%%%%%%%%%%%%%%%%%%%%%%%%%%%%%%%%%%%%%%%%%%%%%%%%%%%%%%%%%
\frame[t]
{
  \frametitle{Hegelian dialectics}
      \begin{block}{Thesis \& anti-thesis}
      
      \begin{itemize}
      	\item History of anything can be explained using Hegelian dichotomy. 
		\item German philosopher Georg Friedrich Hegel explained history as a 
		dialectic between thesis, anti-thesis and resulting synthesis.	
		\item This means that there could emerge a theory first and it could be confronted by 
		an opposing theory. The dialectic between these opposing theories will find a 
		compromise or consensus by assimilating the main aspects of both, in the due course 
		of time. 
	\end{itemize}
   
  \end{block}
}
%%%%%%%%%%%%%%%%%%%%%%%%%%%%%%%%%%%%%%%%%%%%%%%%%%%%%%%%%%%%%%%%%%%%%%%%%%%%%%%%%%%%%%%%%%

%%%%%%%%%%%%%%%%%%%%%%%%%%%%%%%%%%%%%%%%%%%%%%%%%%%%%%%%%%%%%%%%%%%%%%%%%%%%%%%%%%%%%%%%%%
%%%%%%%%%%%%%%%%%%%%%%%%%%%%%%%%%%%%%%%%%%%%%%%%%%%%%%%%%%%%%%%%%%%%%%%%%%%%%%%%%%%%%%%%%%
\subsection{Thesis: one}
%%%%%%%%%%%%%%%%%%%%%%%%%%%%%%%%%%%%%%%%%%%%%%%%%%%%%%%%%%%%%%%%%%%%%%%%%%%%%%%%%%%%%%%%%%
%%%%%%%%%%%%%%%%%%%%%%%%%%%%%%%%%%%%%%%%%%%%%%%%%%%%%%%%%%%%%%%%%%%%%%%%%%%%%%%%%%%%%%%%%%

%%%%%%%%%%%%%%%%%%%%%%%%%%%%%%%%%%%%%%%%%%%%%%%%%%%%%%%%%%%%%%%%%%%%%%%%%%%%%%%%%%%%%%%%%%
\frame[t]
{
  \frametitle{Hegelian dialectics}
  	\framesubtitle{Thesis: one}
      \begin{block}{Two rules}
      
      \begin{itemize}
		\item If we look carefully at the history of computing, we could see these Hegelian 
		cycles of thesis and anti-thesis resulting in a synthesis. 
		\item Computing, or even supercomputing 
		started with one machine getting bigger and gaining speed. That was the thesis. 
		\item Two rules emerged to support this model:
		      \begin{itemize}
		      \item Moore's Law
		      \item Amdahl's Law
		      \end{itemize}		
		\end{itemize}
   
  \end{block}
}
%%%%%%%%%%%%%%%%%%%%%%%%%%%%%%%%%%%%%%%%%%%%%%%%%%%%%%%%%%%%%%%%%%%%%%%%%%%%%%%%%%%%%%%%%%

%%%%%%%%%%%%%%%%%%%%%%%%%%%%%%%%%%%%%%%%%%%%%%%%%%%%%%%%%%%%%%%%%%%%%%%%%%%%%%%%%%%%%%%%%%
%%%%%%%%%%%%%%%%%%%%%%%%%%%%%%%%%%%%%%%%%%%%%%%%%%%%%%%%%%%%%%%%%%%%%%%%%%%%%%%%%%%%%%%%%%
\subsubsection{Moore's law}
%%%%%%%%%%%%%%%%%%%%%%%%%%%%%%%%%%%%%%%%%%%%%%%%%%%%%%%%%%%%%%%%%%%%%%%%%%%%%%%%%%%%%%%%%%
%%%%%%%%%%%%%%%%%%%%%%%%%%%%%%%%%%%%%%%%%%%%%%%%%%%%%%%%%%%%%%%%%%%%%%%%%%%%%%%%%%%%%%%%%%

%%%%%%%%%%%%%%%%%%%%%%%%%%%%%%%%%%%%%%%%%%%%%%%%%%%%%%%%%%%%%%%%%%%%%%%%%%%%%%%%%%%%%%%%%%
\frame[t]
{
  \frametitle{Hegelian dialectics}
      \begin{block}{Moore's law}
        \begin{itemize}
			\item Proposed by Gordon E. Moore (Co-founder of Intel) in 1965. 
			\item In simple terms the law states that processor speeds, or overall 
			processing power for computers will double every two years.  
			\item More specifically it stated that the number of transistors on an 
			affordable CPU would double every two years.
			\item Over roughly 50 years from 1961, the number of transistors 
			doubled approximately every 18 months!
        \end{itemize}		
		
  \end{block}
}
%%%%%%%%%%%%%%%%%%%%%%%%%%%%%%%%%%%%%%%%%%%%%%%%%%%%%%%%%%%%%%%%%%%%%%%%%%%%%%%%%%%%%%%%%%

%%%%%%%%%%%%%%%%%%%%%%%%%%%%%%%%%%%%%%%%%%%%%%%%%%%%%%%%%%%%%%%%%%%%%%%%%%%%%%%%%%%%%%%%%
\frame[t]
{
  \frametitle{Hegelian dialectics}
      \begin{block}{Moore's law in airline industry!}
   		\begin{center}
 			\includegraphics[width=250pt]{fig/ml-flight.pdf}
   		\end{center}  
		
  	\end{block}
}
%%%%%%%%%%%%%%%%%%%%%%%%%%%%%%%%%%%%%%%%%%%%%%%%%%%%%%%%%%%%%%%%%%%%%%%%%%%%%%%%%%%%%%%%%

%%%%%%%%%%%%%%%%%%%%%%%%%%%%%%%%%%%%%%%%%%%%%%%%%%%%%%%%%%%%%%%%%%%%%%%%%%%%%%%%%%%%%%%%%
\frame[t]
{
  \frametitle{Hegelian dialectics}
      \begin{block}{Gordon E. Moore}
   		\begin{center}
 			\includegraphics[width=250pt]{fig/moore-2004.jpg}
   		\end{center}  
		
  	\end{block}
}
%%%%%%%%%%%%%%%%%%%%%%%%%%%%%%%%%%%%%%%%%%%%%%%%%%%%%%%%%%%%%%%%%%%%%%%%%%%%%%%%%%%%%%%%%

%%%%%%%%%%%%%%%%%%%%%%%%%%%%%%%%%%%%%%%%%%%%%%%%%%%%%%%%%%%%%%%%%%%%%%%%%%%%%%%%%%%%%%%%%%
%%%%%%%%%%%%%%%%%%%%%%%%%%%%%%%%%%%%%%%%%%%%%%%%%%%%%%%%%%%%%%%%%%%%%%%%%%%%%%%%%%%%%%%%%%
\subsubsection{Amdahl’s law}
%%%%%%%%%%%%%%%%%%%%%%%%%%%%%%%%%%%%%%%%%%%%%%%%%%%%%%%%%%%%%%%%%%%%%%%%%%%%%%%%%%%%%%%%%%
%%%%%%%%%%%%%%%%%%%%%%%%%%%%%%%%%%%%%%%%%%%%%%%%%%%%%%%%%%%%%%%%%%%%%%%%%%%%%%%%%%%%%%%%%%

%%%%%%%%%%%%%%%%%%%%%%%%%%%%%%%%%%%%%%%%%%%%%%%%%%%%%%%%%%%%%%%%%%%%%%%%%%%%%%%%%%%%%%%%%%
\frame[t]
{
  \frametitle{Hegelian dialectics}
      \begin{block}{Amdahl's law}
        \begin{itemize}
			\item Proposed by Gene Amdahl in his 1967 technical paper titled
			"Validity of the single processor approach to achieving large scale 
			computing capabilities". 
			\item The famous Amdahl's formula below is not in this paper, but only a small
			literal description in the 4th paragraph of the 4 page paper!
			\item "The speedup of a program using multiple processors in parallel 
			computing is limited by the time needed for the sequential fraction of 
			the program."
        \end{itemize}		
		
  \end{block}      
}
%%%%%%%%%%%%%%%%%%%%%%%%%%%%%%%%%%%%%%%%%%%%%%%%%%%%%%%%%%%%%%%%%%%%%%%%%%%%%%%%%%%%%%%%%%

%%%%%%%%%%%%%%%%%%%%%%%%%%%%%%%%%%%%%%%%%%%%%%%%%%%%%%%%%%%%%%%%%%%%%%%%%%%%%%%%%%%%%%%%%
\frame[t]
{
  \frametitle{Hegelian dialectics}
      \begin{block}{Amdahl's equation}
   		\begin{center}
 			\includegraphics[width=100pt]{fig/amdahl-eq.jpg}
   		 \end{center}
   		 
   		 \begin{itemize}
   		 	\item Where:
   		 	   	\begin{itemize}
   		 		 	\item	$P$ is the proportion of a program that can be made parallel 
   		 		 	\item $(1 - P)$ is the proportion that cannot be parallelized 
   		 		 	\item $N$ is the number of processors
   		 		\end{itemize}
			\item	Original idea of Amdahl's law was to show the limitations of 
			parallel computing!
		\end{itemize}
      \end{block}
}
%%%%%%%%%%%%%%%%%%%%%%%%%%%%%%%%%%%%%%%%%%%%%%%%%%%%%%%%%%%%%%%%%%%%%%%%%%%%%%%%%%%%%%%%%

%%%%%%%%%%%%%%%%%%%%%%%%%%%%%%%%%%%%%%%%%%%%%%%%%%%%%%%%%%%%%%%%%%%%%%%%%%%%%%%%%%%%%%%%%
\frame[t]
{
  \frametitle{Hegelian dialectics}
   		\begin{center}
 			\includegraphics[width=180pt]{fig/amdahl-graph.png}
   		\end{center}  
   		 \begin{itemize}
   		 	\item Not even considering the overheads of parallelization
			\item If parallization cannot be done evenly, results will be much worse!
   		 \end{itemize}
}
%%%%%%%%%%%%%%%%%%%%%%%%%%%%%%%%%%%%%%%%%%%%%%%%%%%%%%%%%%%%%%%%%%%%%%%%%%%%%%%%%%%%%%%%%

%%%%%%%%%%%%%%%%%%%%%%%%%%%%%%%%%%%%%%%%%%%%%%%%%%%%%%%%%%%%%%%%%%%%%%%%%%%%%%%%%%%%%%%%%
\frame[t]
{
  \frametitle{Hegelian dialectics}
      \begin{block}{Gene Amdahl}
   		\begin{center}
 			\includegraphics[width=230pt]{fig/amdahl.jpg}
   		\end{center}  
		
  	\end{block}
}
%%%%%%%%%%%%%%%%%%%%%%%%%%%%%%%%%%%%%%%%%%%%%%%%%%%%%%%%%%%%%%%%%%%%%%%%%%%%%%%%%%%%%%%%%

%%%%%%%%%%%%%%%%%%%%%%%%%%%%%%%%%%%%%%%%%%%%%%%%%%%%%%%%%%%%%%%%%%%%%%%%%%%%%%%%%%%%%%%%%%
%%%%%%%%%%%%%%%%%%%%%%%%%%%%%%%%%%%%%%%%%%%%%%%%%%%%%%%%%%%%%%%%%%%%%%%%%%%%%%%%%%%%%%%%%%
\subsection{Anti thesis: many}
%%%%%%%%%%%%%%%%%%%%%%%%%%%%%%%%%%%%%%%%%%%%%%%%%%%%%%%%%%%%%%%%%%%%%%%%%%%%%%%%%%%%%%%%%%
%%%%%%%%%%%%%%%%%%%%%%%%%%%%%%%%%%%%%%%%%%%%%%%%%%%%%%%%%%%%%%%%%%%%%%%%%%%%%%%%%%%%%%%%%%

%%%%%%%%%%%%%%%%%%%%%%%%%%%%%%%%%%%%%%%%%%%%%%%%%%%%%%%%%%%%%%%%%%%%%%%%%%%%%%%%%%%%%%%%%%
%%%%%%%%%%%%%%%%%%%%%%%%%%%%%%%%%%%%%%%%%%%%%%%%%%%%%%%%%%%%%%%%%%%%%%%%%%%%%%%%%%%%%%%%%%
%\subsection{Parallel is natural}
%%%%%%%%%%%%%%%%%%%%%%%%%%%%%%%%%%%%%%%%%%%%%%%%%%%%%%%%%%%%%%%%%%%%%%%%%%%%%%%%%%%%%%%%%%
%%%%%%%%%%%%%%%%%%%%%%%%%%%%%%%%%%%%%%%%%%%%%%%%%%%%%%%%%%%%%%%%%%%%%%%%%%%%%%%%%%%%%%%%%%
%%%%%%%%%%%%%%%%%%%%%%%%%%%%%%%%%%%%%%%%%%%%%%%%%%%%%%%%%%%%%%%%%%%%%%%%%%%%%%%%%%%%%%%%%%
\frame[t]
{
  \frametitle{Hegelian dialectics}
    \framesubtitle{Anti-thesis: many}
      \begin{block}{Rules that go wrong!}
		\begin{itemize}
   		 	\item Amdahl's law actually revealed the possibility of parallel computing:
   		 	\begin{itemize}
   		 	   	\item There is actual increase in speed, if the algorithm is parallelizable.
   		 		\item There is practically no other way to increase speed in many cases, 
   		 		even if this is not the most efficient way to do it.	
   		 	\end{itemize}
			\item Collapse of Moore's law is in the vicinity:
   		 	\begin{itemize}
   		 		\item  "In about 10 years or so, we will see the collapse 
      			of Moore's Law" says Physicist Michio Kaku, a professor of theoretical 
      			physics at City University of New York (2013).   		 		
      			\item   Because of the heat and leakage associated with silicon based 
      			transistors. 
   		 	\end{itemize}			
			
   		 \end{itemize}  
  \end{block}
   	\note {This slide is intentionally left blank.} 
}
%%%%%%%%%%%%%%%%%%%%%%%%%%%%%%%%%%%%%%%%%%%%%%%%%%%%%%%%%%%%%%%%%%%%%%

%%%%%%%%%%%%%%%%%%%%%%%%%%%%%%%%%%%%%%%%%%%%%%%%%%%%%%%%%%%%%%%%%%%%%%%%%%%%%%%%%%%%%%%%%%
%%%%%%%%%%%%%%%%%%%%%%%%%%%%%%%%%%%%%%%%%%%%%%%%%%%%%%%%%%%%%%%%%%%%%%%%%%%%%%%%%%%%%%%%%%
\section{Parallel computing}
%%%%%%%%%%%%%%%%%%%%%%%%%%%%%%%%%%%%%%%%%%%%%%%%%%%%%%%%%%%%%%%%%%%%%%%%%%%%%%%%%%%%%%%%%%
%%%%%%%%%%%%%%%%%%%%%%%%%%%%%%%%%%%%%%%%%%%%%%%%%%%%%%%%%%%%%%%%%%%%%%%%%%%%%%%%%%%%%%%%%%

%%%%%%%%%%%%%%%%%%%%%%%%%%%%%%%%%%%%%%%%%%%%%%%%%%%%%%%%%%%%%%%%%%%%%%%%%%%%%%%%%%%%%%%%%%
\frame[t]
{
  \frametitle{Parallel computing}
    \begin{block}{The anti-thesis}
      	\begin{center}
 			\includegraphics[width=230pt]{fig/hegelian-1.png}
   		\end{center}     
  \end{block}
}
%%%%%%%%%%%%%%%%%%%%%%%%%%%%%%%%%%%%%%%%%%%%%%%%%%%%%%%%%%%%%%%%%%%%%%%%%%%%%%%%%%%%%%%%%%


%%%%%%%%%%%%%%%%%%%%%%%%%%%%%%%%%%%%%%%%%%%%%%%%%%%%%%%%%%%%%%%%%%%%%%
\frame[t]
{
  \frametitle{Parallel computing}
    \begin{block}{Background}
      Parallel computing emerged as an anti-thesis. Factors that have accelerated 
    	\begin{itemize}
   		 	\item The arrival of cheap commodity computing hardware 
   		 	\item The theoretical possibility of achieving the same or even greater speeds 
   		 	of serial HPC computing by harnessing the distributed computing capabilities 
   		 	of commodity hardware
			\item Recognition that the nature itself is parallel, however complex it may 
			appear. For example:
				\begin{itemize}
   		 			\item Parallel genome sequencing algorithms
   		 			\item Parallel Monte Carlo algorithms
   		 		\end{itemize}
		\end{itemize} 
  	\end{block}
}
%%%%%%%%%%%%%%%%%%%%%%%%%%%%%%%%%%%%%%%%%%%%%%%%%%%%%%%%%%%%%%%%%%%%%%%%%%%%%%%%%%%%%%%%%%


%%%%%%%%%%%%%%%%%%%%%%%%%%%%%%%%%%%%%%%%%%%%%%%%%%%%%%%%%%%%%%%%%%%%%%%%%%%%%%%%%%%%%%%%%%
%%%%%%%%%%%%%%%%%%%%%%%%%%%%%%%%%%%%%%%%%%%%%%%%%%%%%%%%%%%%%%%%%%%%%%%%%%%%%%%%%%%%%%%%%%
\subsection{Key concepts}
%%%%%%%%%%%%%%%%%%%%%%%%%%%%%%%%%%%%%%%%%%%%%%%%%%%%%%%%%%%%%%%%%%%%%%%%%%%%%%%%%%%%%%%%%%

%%%%%%%%%%%%%%%%%%%%%%%%%%%%%%%%%%%%%%%%%%%%%%%%%%%%%%%%%%%%%%%%%%%%%%%%%%%%%%%%%%%%%%%%%%
%%%%%%%%%%%%%%%%%%%%%%%%%%%%%%%%%%%%%%%%%%%%%%%%%%%%%%%%%%%%%%%%%%%%%%%%%%%%%%%%%%%%%%%%%%
\frame[t]
{
  \frametitle{Parallel computing}
      \begin{block}{Definition}
      \begin{itemize}
		\item Parallel computing is the art of breaking up a big chunk of serial computation into smaller 
		atomic units which can be done in parallel. 
		\item "It is the simultaneous use of multiple compute resources to 
		solve a computational problem" \footnotemark.
      \end{itemize}

   \footnotetext[1]{Ref: Lawrence Livermore National Laboratory, US}
   	\end{block}
}
%%%%%%%%%%%%%%%%%%%%%%%%%%%%%%%%%%%%%%%%%%%%%%%%%%%%%%%%%%%%%%%%%%%%%%%%%%%%%%%%%%%%%%%%%%


%%%%%%%%%%%%%%%%%%%%%%%%%%%%%%%%%%%%%%%%%%%%%%%%%%%%%%%%%%%%%%%%%%%%%%%%%%%%%%%%%%%%%%%%%%
\frame[t]
{
  \frametitle{Parallel computing}
      \begin{block}{Key concepts}
   		\begin{itemize}%[<+-| alert@+>]
			\item A cluster is a network of computers, sometimes called nodes or hosts.
			\item Each computer has several processors.
			\item Each processor has several cores.
			\item A core does the computing.
			\item If an application uses more than one core, it runs faster on a cluster.
   		\end{itemize}
  	\end{block}
}
%%%%%%%%%%%%%%%%%%%%%%%%%%%%%%%%%%%%%%%%%%%%%%%%%%%%%%%%%%%%%%%%%%%%%%%%%%%%%%%%%%%%%%%%%%

%%%%%%%%%%%%%%%%%%%%%%%%%%%%%%%%%%%%%%%%%%%%%%%%%%%%%%%%%%%%%%%%%%%%%%%%%%%%%%%%%%%%%%%%%%
\frame[t]
{
  \frametitle{Parallel computing}
   \begin{block}{A node overview}
   \begin{center}
 		\includegraphics[width=270pt]{NeSI_img/node_view.png}
   	\end{center}
  	\end{block}
}

%%%%%%%%%%%%%%%%%%%%%%%%%%%%%%%%%%%%%%%%%%%%%%%%%%%%%%%%%%%%%%%%%%%%%%%%%%%%%%%%%%%%%%%%%%
%%%%%%%%%%%%%%%%%%%%%%%%%%%%%%%%%%%%%%%%%%%%%%%%%%%%%%%%%%%%%%%%%%%%%%%%%%%%%%%%%%%%%%%%%%
\frame[t]
{
  \frametitle{Parallel computing}
  \framesubtitle{Principle}  
      \begin{itemize}
    	\item Breaking down of a problem into discrete parts that can be solved concurrently
    	\item Each part is further broken down to a series of instructions
     	\item Instructions from each part execute simultaneously on different processors
    \end{itemize}  
          	\begin{center}
 			\includegraphics[width=200pt]{fig/parallel-computing.png}
   		\end{center}   
}

%%%%%%%%%%%%%%%%%%%%%%%%%%%%%%%%%%%%%%%%%%%%%%%%%%%%%%%%%%%%%%%%%%%%%%%%%%%%%%%%%%%%%%%%%%
%%%%%%%%%%%%%%%%%%%%%%%%%%%%%%%%%%%%%%%%%%%%%%%%%%%%%%%%%%%%%%%%%%%%%%%%%%%%%%%%%%%%%%%%%%
\frame[t]
{
  \frametitle{Evolution of parallel computing}
     \begin{block}{Grid \& SOA}	      
        \begin{itemize}
           	\item Grid Computing
   			 	\begin{itemize}
   					\item It is the distributed computing model which orchestrates 
   					distributed computing resources of different academic and research 
   					institutions to work like a single, unified computing system. 
   			   	\end{itemize}		
   			\item Service Oriented Approach (SOA)
   			 	\begin{itemize}
   					\item SOA is a programming paradigm based on the idea of 
   					composing applications by invoking network-available services 
   					to accomplish some tasks, mainly adopted by enterprise circles.
   					\item This was adopted mainly by business and enterprise community. 
   				\end{itemize}

   		   \end{itemize}
   	\end{block}
}
%%%%%%%%%%%%%%%%%%%%%%%%%%%%%%%%%%%%%%%%%%%%%%%%%%%%%%%%%%%%%%%%%%%%%%%%%%%%%%%%%%%%%%%%%%

%%%%%%%%%%%%%%%%%%%%%%%%%%%%%%%%%%%%%%%%%%%%%%%%%%%%%%%%%%%%%%%%%%%%%%%%%%%%%%%%%%%%%%%%%%
\frame[t]
{
  \frametitle{Evolution of parallel computing}
  \framesubtitle{High Performance Computing}
        \begin{itemize}
   			\item TOP500 list of 2002 reported that 20 percent of HPC 
			installations as "clusters"- parallel computing emerged 
			as a serious platform for HPC.          
			\item TOP500 list of 2013 reports that "cluster" share is about 80 percent.
        \end{itemize}   		 
   		\begin{center}
 			\includegraphics[width=220pt]{fig/TOP50201311_Poster.png}
   		\end{center} 

}
%%%%%%%%%%%%%%%%%%%%%%%%%%%%%%%%%%%%%%%%%%%%%%%%%%%%%%%%%%%%%%%%%%%%%%%%%%%%%%%%%%%%%%%%%%

%%%%%%%%%%%%%%%%%%%%%%%%%%%%%%%%%%%%%%%%%%%%%%%%%%%%%%%%%%%%%%%%%%%%%%%%%%%%%%%%%%%%%%%%%%
\frame[t]
{
  \frametitle{Hegelian cycles of parallel computing}

      	\begin{center}
 			\includegraphics[width=230pt]{fig/hegelian-2.png}
   		\end{center}     

}
%%%%%%%%%%%%%%%%%%%%%%%%%%%%%%%%%%%%%%%%%%%%%%%%%%%%%%%%%%%%%%%%%%%%%%%%%%%%%%%%%%%%%%%%%%

%%%%%%%%%%%%%%%%%%%%%%%%%%%%%%%%%%%%%%%%%%%%%%%%%%%%%%%%%%%%%%%%%%%%%%%%%%%%%%%%%%%%%%%%%%
\frame[t]
{
  \frametitle{Hegelian synthesis}
      \begin{block}{NeSI like systems}
      	\begin{itemize}
   			\item Mixing grid computing with high performance computing
   			\item Aiming to cater to research and academic community
         \end{itemize}
  	\end{block}
    \begin{block}{Cloud providers}
    	\begin{itemize}
   			\item Merging serviced oriented computing with grid concepts
   			\item Providing services to enterprises and small businesses
   			    \begin{itemize}
   			    	\item  It uses virtualization techniques to orchestrate storage, memory and 
   			    	network resources of geographically distributed data centers as a unified 
   			    	unit.
   			    	\item It is made available to customers on-demand with apparent elasticity.
   			    \end{itemize}
         \end{itemize}
  	\end{block}
}
%%%%%%%%%%%%%%%%%%%%%%%%%%%%%%%%%%%%%%%%%%%%%%%%%%%%%%%%%%%%%%%%%%%%%%%%%%%%%%%%%%%%%%%%%%

%%%%%%%%%%%%%%%%%%%%%%%%%%%%%%%%%%%%%%%%%%%%%%%%%%%%%%%%%%%%%%%%%%%%%%%%%%%%%%%%%%%%%%%%%%
%%%%%%%%%%%%%%%%%%%%%%%%%%%%%%%%%%%%%%%%%%%%%%%%%%%%%%%%%%%%%%%%%%%%%%%%%%%%%%%%%%%%%%%%%%
\frame[t]
{
  \frametitle{Synthesis}

      	\begin{center}
 			\includegraphics[width=230pt]{fig/hegelian-3.png}
   		\end{center}     
}
%%%%%%%%%%%%%%%%%%%%%%%%%%%%%%%%%%%%%%%%%%%%%%%%%%%%%%%%%%%%%%%%%%%%%%%%%%%%%%%%%%%%%%%%%%

%%%%%%%%%%%%%%%%%%%%%%%%%%%%%%%%%%%%%%%%%%%%%%%%%%%%%%%%%%%%%%%%%%%%%%%%%%%%%%%%%%%%%%%%%%
%\subsection{NeSI Systems}
%%%%%%%%%%%%%%%%%%%%%%%%%%%%%%%%%%%%%%%%%%%%%%%%%%%%%%%%%%%%%%%%%%%%%%%%%%%%%%%%%%%%%%%%%%
%%%%%%%%%%%%%%%%%%%%%%%%%%%%%%%%%%%%%%%%%%%%%%%%%%%%%%%%%%%%%%%%%%%%%%%%%%%%%%%%%%%%%%%%%%

%%%%%%%%%%%%%%%%%%%%%%%%%%%%%%%%%%%%%%%%%%%%%%%%%%%%%%%%%%%%%%%%%%%%%%%%%%%%%%%%%%%%%%%%%%
\frame[t]
{
  \frametitle{NeSI Systems}
      \begin{block}{Facilites}   
      	NeSI HPC resources cater to research and academic community of New Zealand and 
      	are spread around three major facilities:
		\begin{itemize}
			\item BlueFern Supercomputing Center at the University of Canterbury, Christchurch
			\item NIWA Supercomputing Center, Wellington
			\item CeR Supercomputing Center at the University of Auckland. 
		\end{itemize}	
  There is a single \textbf{grid} interface called \textbf{Grisu} to access both BlueFern 
  and CeR (NeSI Pan) clusters.
  \end{block}
}
%%%%%%%%%%%%%%%%%%%%%%%%%%%%%%%%%%%%%%%%%%%%%%%%%%%%%%%%%%%%%%%%%%%%%%%%%%%%%%%%%%%%%%%%%%
%%%%%%%%%%%%%%%%%%%%%%%%%%%%%%%%%%%%%%%%%%%%%%%%%%%%%%%%%%%%%%%%%%%%%%%%%%%%%%%%%%%%%%%%%%
\frame[t]
{
  \frametitle{NeSI Systems}
      \begin{block}{Available HPC architectures}
		NeSI provides several kind of HPC architectures and solutions to cater for various needs.
     	\begin{itemize}
     		\item Bluegene P
	    	\item Power6 and Power7
   	    	\item Intel Westmere \& SandyBridge
			\item Kepler and Fermi GPU servers
			\item Intel Xeon Phi Co-Processor
   		\end{itemize}	
  \end{block}
}

%%%%%%%%%%%%%%%%%%%%%%%%%%%%%%%%%%%%%%%%%%%%%%%%%%%%%%%%%%%%%%%%%%%%%%%%%%%%%%%%%%%%%%%%%%



%%%%%%%%%%%%%%%%%%%%%%%%%%%%%%%%%%%%%%%%%%%%%%%%%%%%%%%%%%%%%%%%%%%%%%%%%%%%%%%%%%%%%%%%%%
\frame[t]
{
  \frametitle{NeSI Systems}
      \begin{block}{Available bioinformatics applications}
     	Many general purpose scientific applications are already available in NeSI systems.
     	\begin{itemize}
     		\item \textbf{Velvet:} Sequence assembler for very short reads
	    	\item \textbf{Bowtie:} Aligns short reads to a reference genome
   	    	\item \textbf{BLAST:} Searches for regions of similarity between biological sequences
			\item \textbf{Hmmer:} Protein sequence search and alignment with Hidden Markov Models 
			\item \textbf{Muscle:} Multiple sequence aligner
			\item \textbf{PhyML:} Maximum likelihood phylogenies
			\item \textbf{MrBayes:}  Bayesian inference and model choice across a wide range 
			of phylogenetic and evolutionary models
   		\end{itemize}	
  \end{block}
}

%%%%%%%%%%%%%%%%%%%%%%%%%%%%%%%%%%%%%%%%%%%%%%%%%%%%%%%%%%%%%%%%%%%%%%%%%%%%%%%%%%%%%%%%%%

%%%%%%%%%%%%%%%%%%%%%%%%%%%%%%%%%%%%%%%%%%%%%%%%%%%%%%%%%%%%%%%%%%%%%%%%%%%%%%%%%%%%%%%%%%
\frame[t]
{
  \frametitle{NeSI Systems}
  \framesubtitle{Performance analysis}
  \begin{block}{PhyML Case Study}
  \textbf{PhyML} is a software that estimates maximum likelihood phylogenies from alignments of nucleotide or amino acid sequences. The main strength of PhyML lies in the large number of substitution models coupled to various options to search the space of phylogenetic tree topologies, going from very fast and efficient methods to slower but generally more accurate approaches. \\
The right compilers and optimization options for an specific architecture can increase the performance quite a lot!
  \end{block}
}
%%%%%%%%%%%%%%%%%%%%%%%%%%%%%%%%%%%%%%%%%%%%%%%%%%%%%%%%%%%%%%%%%%%%%%%%%%%%%%%%%%%%%%%%%%
%%%%%%%%%%%%%%%%%%%%%%%%%%%%%%%%%%%%%%%%%%%%%%%%%%%%%%%%%%%%%%%%%%%%%%%%%%%%%%%%%%%%%%%%%%
\frame[t]
{
  \frametitle{NeSI Systems}
  \framesubtitle{Performance analysis}
      \begin{block}{PhyML Case Study : Speed Up}
\begin{center}
 \includegraphics[width=\textwidth]{fig/speedup-phyml.png}
\end{center}
  \end{block}
}
%%%%%%%%%%%%%%%%%%%%%%%%%%%%%%%%%%%%%%%%%%%%%%%%%%%%%%%%%%%%%%%%%%%%%%%%%%%%%%%%%%%%%%%%%%


%%%%%%%%%%%%%%%%%%%%%%%%%%%%%%%%%%%%%%%%%%%%%%%%%%%%%%%%%%%%%%%%%%%%%%%%%%%%%%%%%%%%%%%%%%
%%%%%%%%%%%%%%%%%%%%%%%%%%%%%%%%%%%%%%%%%%%%%%%%%%%%%%%%%%%%%%%%%%%%%%%%%%%%%%%%%%%%%%%%%%
\section{Parallel programming}
%%%%%%%%%%%%%%%%%%%%%%%%%%%%%%%%%%%%%%%%%%%%%%%%%%%%%%%%%%%%%%%%%%%%%%%%%%%%%%%%%%%%%%%%%%
%%%%%%%%%%%%%%%%%%%%%%%%%%%%%%%%%%%%%%%%%%%%%%%%%%%%%%%%%%%%%%%%%%%%%%%%%%%%%%%%%%%%%%%%%%

%%%%%%%%%%%%%%%%%%%%%%%%%%%%%%%%%%%%%%%%%%%%%%%%%%%%%%%%%%%%%%%%%%%%%%%%%%%%%%%%%%%%%%%%%%
\frame[t]
{
  \frametitle{Parallel programming}
      \begin{block}{Rationale}
		\begin{itemize}
     		\item Is there a rationale for writing your own parallel codes
     		when we already got general purpose scientific applications?
     		\item General purpose scientific applications are good for research. 
     			\begin{itemize}
     				\item 80 percent of NeSI cluster usage is by those applications at the moment.
     				\item They are easy to customise and use. 
     			\end{itemize}
     		\item However as the research complexity increases and its scope is refined
     		general purpose scientific applications may not cater to all the needs of researchers.
     	\end{itemize}
  \end{block}
}
%%%%%%%%%%%%%%%%%%%%%%%%%%%%%%%%%%%%%%%%%%%%%%%%%%%%%%%%%%%%%%%%%%%%%%%%%%%%%%%%%%%%%%%%%%

%%%%%%%%%%%%%%%%%%%%%%%%%%%%%%%%%%%%%%%%%%%%%%%%%%%%%%%%%%%%%%%%%%%%%%%%%%%%%%%%%%%%%%%%%%
\frame[t]
{
  \frametitle{Parallel problems}
  \framesubtitle{Case Study: N Body Problem} 
  	\begin{center} 
      	\includegraphics[width=230pt]{fig/solar-system.jpg}	
	\end{center}  
	\begin{center}  
		Gravitational n-body problem where: \\
		Force is a function of mass and acceleration ($f = m*a$)
	\end{center}  
}
%%%%%%%%%%%%%%%%%%%%%%%%%%%%%%%%%%%%%%%%%%%%%%%%%%%%%%%%%%%%%%%%%%%%%%%%%%%%%%%%%%%%%%%%%%

%%%%%%%%%%%%%%%%%%%%%%%%%%%%%%%%%%%%%%%%%%%%%%%%%%%%%%%%%%%%%%%%%%%%%%%%%%%%%%%%%%%%%%%%%%
\frame[t]
{
  \frametitle{Parallel programming}
  \framesubtitle{Case Study: N Body Problem}
    	\begin{figure}
   			\includegraphics[width=0.45\textwidth]{fig/john-chambers.png}
   			\hfill
   			\includegraphics[width=0.40\textwidth]{fig/joseph-hahn.png}
      	\end{figure}  
	\begin{itemize}
     		\item Mercury: N body simulator developed
     		by  Dr John Chamber at NASA Ames Research Center, California
     		\item Dr Joseph M. Hahn of Space Science Institue, Colorado had to come up with a 
     		varient Mercury to take care of additional drag forces that drive planet migration!
    \end{itemize}    		

}
%%%%%%%%%%%%%%%%%%%%%%%%%%%%%%%%%%%%%%%%%%%%%%%%%%%%%%%%%%%%%%%%%%%%%%%%%%%%%%%%%%%%%%%%%%


%%%%%%%%%%%%%%%%%%%%%%%%%%%%%%%%%%%%%%%%%%%%%%%%%%%%%%%%%%%%%%%%%%%%%%%%%%%%%%%%%%%%%%%%%%
\frame[t]
{
  \frametitle{Parallel programming}
  \framesubtitle{Case Study: N Body Problem}
      
    	\begin{center}
   			\includegraphics[width=180pt]{fig/philip-sharp.png}
      	\end{center} 
      	
		\begin{itemize}
     		\item Philip Sharp at University of Auckland Started 
     		implementing Mercury n-body integration on GPUs.
     		\item Then he found out that he has been experimenting with more robust 
     		versions of those techniques.
     		\item He is now rebuilding the n-body integrator from scratch. 
     	\end{itemize}

}
%%%%%%%%%%%%%%%%%%%%%%%%%%%%%%%%%%%%%%%%%%%%%%%%%%%%%%%%%%%%%%%%%%%%%%%%%%%%%%%%%%%%%%%%%%

%%%%%%%%%%%%%%%%%%%%%%%%%%%%%%%%%%%%%%%%%%%%%%%%%%%%%%%%%%%%%%%%%%%%%%%%%%%%%%%%%%%%%%%%%%
%%%%%%%%%%%%%%%%%%%%%%%%%%%%%%%%%%%%%%%%%%%%%%%%%%%%%%%%%%%%%%%%%%%%%%%%%%%%%%%%%%%%%%%%%%
\subsection{Parallel decomposition}
%%%%%%%%%%%%%%%%%%%%%%%%%%%%%%%%%%%%%%%%%%%%%%%%%%%%%%%%%%%%%%%%%%%%%%%%%%%%%%%%%%%%%%%%%%
%%%%%%%%%%%%%%%%%%%%%%%%%%%%%%%%%%%%%%%%%%%%%%%%%%%%%%%%%%%%%%%%%%%%%%%%%%%%%%%%%%%%%%%%%%


%%%%%%%%%%%%%%%%%%%%%%%%%%%%%%%%%%%%%%%%%%%%%%%%%%%%%%%%%%%%%%%%%%%%%%%%%%%%%%%%%%%%%%%%%%
%%%%%%%%%%%%%%%%%%%%%%%%%%%%%%%%%%%%%%%%%%%%%%%%%%%%%%%%%%%%%%%%%%%%%%%%%%%%%%%%%%%%%%%%%%
\subsubsection{N body problem}
%%%%%%%%%%%%%%%%%%%%%%%%%%%%%%%%%%%%%%%%%%%%%%%%%%%%%%%%%%%%%%%%%%%%%%%%%%%%%%%%%%%%%%%%%%
%%%%%%%%%%%%%%%%%%%%%%%%%%%%%%%%%%%%%%%%%%%%%%%%%%%%%%%%%%%%%%%%%%%%%%%%%%%%%%%%%%%%%%%%%%

%%%%%%%%%%%%%%%%%%%%%%%%%%%%%%%%%%%%%%%%%%%%%%%%%%%%%%%%%%%%%%%%%%%%%%%%%%%%%%%%%%%%%%%%%%
\frame[t]
{
  \frametitle{Parallel programming}
    \framesubtitle{Case Study: N Body Problem}
    We need to look for hotspots and bottlenecks to parallelize an algorithm. 
      \begin{block}{Hotspots}
      	
		\begin{itemize}  
     		\item Hotspots are the areas where we have massive iterative works, which 
     		can be parallelized into optimally smaller units that are 
     		independent of each other. 
     		\item Those units of work are called tasks and such algorithms can be 
     		called "embarrassingly parallel" in nature.
      	\end{itemize}
   
  \end{block}
}
%%%%%%%%%%%%%%%%%%%%%%%%%%%%%%%%%%%%%%%%%%%%%%%%%%%%%%%%%%%%%%%%%%%%%%%%%%%%%%%%%%%%%%%%%%

%%%%%%%%%%%%%%%%%%%%%%%%%%%%%%%%%%%%%%%%%%%%%%%%%%%%%%%%%%%%%%%%%%%%%%%%%%%%%%%%%%%%%%%%%%
\frame[t]
{
  \frametitle{Parallel programming}
    \framesubtitle{Case Study: N Body Problem}
      \begin{block}{Bottlenecks}
      	
		\begin{itemize}  
     		\item Bottlenecks are the places where computations becomes inter-dependent. 
     		Usually it is where there will be a need to synchronise the data before we 
     		proceed to do another set of parallel tasks. 
     		\item If an algorithm has got too many inter-dependent tasks, it will be called
     		a fine-grained algorithm and may not able to parallelize efficiently. 
      	\end{itemize}
   
  \end{block}
}
%%%%%%%%%%%%%%%%%%%%%%%%%%%%%%%%%%%%%%%%%%%%%%%%%%%%%%%%%%%%%%%%%%%%%%%%%%%%%%%%%%%%%%%%%%


%%%%%%%%%%%%%%%%%%%%%%%%%%%%%%%%%%%%%%%%%%%%%%%%%%%%%%%%%%%%%%%%%%%%%%%%%%%%%%%%%%%%%%%%%%
\begin{frame}{Direct n-body problem}
  \framesubtitle{Governing equations}
	\begin{equation*}
	\textbf{f}_{ij} = G \frac{m_i m_j}{||\textbf{r}_{ij}||^2} \cdot \frac{\textbf{r}_{ij}}{||\textbf{r}_{ij}||}
	\end{equation*}
	\begin{equation*}
	\textbf{F}_i = \sum_{1 \le j \le N} \textbf{f}_{ij} \;\; ,\; j \ne i
	\end{equation*}
	\begin{equation*}
	\textbf{F}_i = m_i \textbf{a}_i
	\end{equation*}
	\begin{equation*}
	\textbf{p}_i = \iint \textbf{a}_i\,\textrm{d}t
	\end{equation*}
Where:
	\begin{itemize}
	 	\item Force (F) is function of mass and acceleration and
	 	\item Position (p) is a function of velocity/acceleration and time.
	\end{itemize}	 	
	
\end{frame}
%%%%%%%%%%%%%%%%%%%%%%%%%%%%%%%%%%%%%%%%%%%%%%%%%%%%%%%%%%%%%%%%%%%%%%%%%%%%%%%%%%%%%%%%%%

%%%%%%%%%%%%%%%%%%%%%%%%%%%%%%%%%%%%%%%%%%%%%%%%%%%%%%%%%%%%%%%%%%%%%%%%%%%%%%%%%%%%%%%%%%

\begin{frame}[fragile]
\frametitle{Direct n-body problem}
  \framesubtitle{Pseudocode}
	\begin{algorithm}[H]
	\begin{algorithmic}[1]


\STATE Input initial positons and velocities of particles
\WHILE {each simulation step}
	\FOR {each particle}\\
		\STATE Compute total force on the particle \\
	\ENDFOR
	
	\FOR {each particle}
		\STATE Compute velocity and position of the particle\\
	\ENDFOR	
\STATE  Output new velocity and position of particles\\
\ENDWHILE
\STATE  Output final velocity and position of particles\\

\end{algorithmic}
%\caption{Pseudocode for N body problem }
\label{alg:seq}
\end{algorithm}
\end{frame}


%%%%%%%%%%%%%%%%%%%%%%%%%%%%%%%%%%%%%%%%%%%%%%%%%%%%%%%%%%%%%%%%%%%%%%%%%%%%%%%%%%%%%%%%%%


%%%%%%%%%%%%%%%%%%%%%%%%%%%%%%%%%%%%%%%%%%%%%%%%%%%%%%%%%%%%%%%%%%%%%%%%%%%%%%%%%%%%%%%%%%
\frame[t]
{
  \frametitle{Direct n-body problem}
      \begin{block}{Parallel decomposition}
      There are two major decomposition techniques:
      	\begin{itemize}
      	\item Domain/data decomposition
		  \begin{itemize}
				\item It forms the foundation for most parallel algorithms 
				\item Here we assign different subset of data to different processors
				and do the same or similar computation over it. 
			\end{itemize}
      	\item Functional/task decomposition
		  \begin{itemize}
				\item It is an alternate way where we assign different 
				functions to different processors
				\item When algorithms does not have any obvious data structure that can be decomposed.
				\item It could the case of a single task at the root of a 
				tree creates new tasks for each subtree, based on the mode of computation
				rather than the structure of the data.
			\end{itemize}
      	\end{itemize}
   
  \end{block}
}
%%%%%%%%%%%%%%%%%%%%%%%%%%%%%%%%%%%%%%%%%%%%%%%%%%%%%%%%%%%%%%%%%%%%%%%%%%%%%%%%%%%%%%%%%%

%%%%%%%%%%%%%%%%%%%%%%%%%%%%%%%%%%%%%%%%%%%%%%%%%%%%%%%%%%%%%%%%%%%%%%%%%%%%%%%%%%%%%%%%%%
%%%%%%%%%%%%%%%%%%%%%%%%%%%%%%%%%%%%%%%%%%%%%%%%%%%%%%%%%%%%%%%%%%%%%%%%%%%%%%%%%%%%%%%%%%
%\subsubsection{Data decomposition}
%%%%%%%%%%%%%%%%%%%%%%%%%%%%%%%%%%%%%%%%%%%%%%%%%%%%%%%%%%%%%%%%%%%%%%%%%%%%%%%%%%%%%%%%%%
%%%%%%%%%%%%%%%%%%%%%%%%%%%%%%%%%%%%%%%%%%%%%%%%%%%%%%%%%%%%%%%%%%%%%%%%%%%%%%%%%%%%%%%%%%

%%%%%%%%%%%%%%%%%%%%%%%%%%%%%%%%%%%%%%%%%%%%%%%%%%%%%%%%%%%%%%%%%%%%%%%%%%%%%%%%%%%%%%%%%%
\frame[t]
{
  \frametitle{Direct n-body problem}
      \begin{block}{Parallel decomposition}
      		\begin{itemize}
				\item We can visualize bodies in direct n-body problem as members of 
				an array or a matrix (2 dimensional array).
				\item This data can be divided into subsets: it is a case of domain/data
				parallelism.

			\end{itemize}
   
  \end{block}
}
%%%%%%%%%%%%%%%%%%%%%%%%%%%%%%%%%%%%%%%%%%%%%%%%%%%%%%%%%%%%%%%%%%%%%%%%%%%%%%%%%%%%%%%%%

%%%%%%%%%%%%%%%%%%%%%%%%%%%%%%%%%%%%%%%%%%%%%%%%%%%%%%%%%%%%%%%%%%%%%%%%%%%%%%%%%%%%%%%


%%%%%%%%%%%%%%%%%%%%%%%%%%%%%%%%%%%%%%%%%%%%%%%%%%%%%%%%%%%%%%%%%%%%%%%%%%%%%%%%%%%%%%%%%%
\frame[t]
{
  \frametitle{Direct n-body problem}
      \begin{block}{Data decomposition as an array}
        \begin{figure}
   			\includegraphics[width=0.475\textwidth]{fig/grid-1d.pdf}
   			\hfill
   			\includegraphics[width=0.475\textwidth]{fig/grid-1d-decomp.pdf}
      	\end{figure}   
  \end{block}
}
%%%%%%%%%%%%%%%%%%%%%%%%%%%%%%%%%%%%%%%%%%%%%%%%%%%%%%%%%%%%%%%%%%%%%%%%%%%%%%%%%%%%%%%%%%

%%%%%%%%%%%%%%%%%%%%%%%%%%%%%%%%%%%%%%%%%%%%%%%%%%%%%%%%%%%%%%%%%%%%%%%%%%%%%%%%%%%%%%%%%%
\frame[t]
{
  \frametitle{Direct n-body problem}
      \begin{block}{Data decomposition as a matrix}
    	\begin{figure}
   			\includegraphics[width=0.475\textwidth]{fig/grid-2d.pdf}
   			\hfill
   			\includegraphics[width=0.475\textwidth]{fig/grid-2d-decomp.pdf}
      	\end{figure}      
  \end{block}
}
%%%%%%%%%%%%%%%%%%%%%%%%%%%%%%%%%%%%%%%%%%%%%%%%%%%%%%%%%%%%%%%%%%%%%%%%%%%%%%%%%%%%%%%%%%

%%%%%%%%%%%%%%%%%%%%%%%%%%%%%%%%%%%%%%%%%%%%%%%%%%%%%%%%%%%%%%%%%%%%%%%%%%%%%%%%%%%%%%%%%%
\frame[t]
{
  \frametitle{Direct n-body problem}
      \begin{block}{Parallel decomposition}
      		\begin{itemize}
      			\item The algorithm is reduced to parallel matrix-vector operations
				(Note: Discussion on OpenMP will use an example of matrix multiplication) 
				\item However all nodes need to have access to all the data in this model,
				which could put constraint on memory or latency.
				\item There are other n-body algorithms to mitigate this issue, where 
				bodies are geographically grouped together to consider it as a large 
				single body (Example: Barnes–Hut method).
				\item All data matrix problems cannot be divided as we did in direct 
				n-body method. 
			\end{itemize}
   
  \end{block}
}
%%%%%%%%%%%%%%%%%%%%%%%%%%%%%%%%%%%%%%%%%%%%%%%%%%%%%%%%%%%%%%%%%%%%%%%%%%%%%%%%%%%%%%%%%

%%%%%%%%%%%%%%%%%%%%%%%%%%%%%%%%%%%%%%%%%%%%%%%%%%%%%%%%%%%%%%%%%%%%%%%%%%%%%%%%%%%%%%%%%%
\frame[t]
{
  \frametitle{N-body problem}
  \framesubtitle{Barnes–Hut method}
    \begin{block}{Parallel decomposition}
         \begin{center}
 			\includegraphics[width=130pt]{fig/barnes-hut.png}
   		 \end{center} 
   	It reduces the complexity of problem from $O(n^2)$to $O(n log n)$ 
   	\end{block} 
}
%%%%%%%%%%%%%%%%%%%%%%%%%%%%%%%%%%%%%%%%%%%%%%%%%%%%%%%%%%%%%%%%%%%%%%%%%%%%%%%%%%%%%%%%%%



%%%%%%%%%%%%%%%%%%%%%%%%%%%%%%%%%%%%%%%%%%%%%%%%%%%%%%%%%%%%%%%%%%%%%%%%%%%%%%%%%%%%%%%%%%
\frame[t]
{
  \frametitle{Parallel decomposition}
  	\framesubtitle{Heat diffusion problem}
      \begin{figure}%[htb]
         \begin{center}
 			\includegraphics[width=150pt]{fig/heat-diffusion-llnl.png}\\
   			\caption {Heat diffusion on a 2D plate}  
   		 \end{center}  
   		 For updating the cells, we need all the neighbours of all the cells. 
   	 \end{figure}
}
%%%%%%%%%%%%%%%%%%%%%%%%%%%%%%%%%%%%%%%%%%%%%%%%%%%%%%%%%%%%%%%%%%%%%%%%%%%%%%%%%%%%%%%%%%


%%%%%%%%%%%%%%%%%%%%%%%%%%%%%%%%%%%%%%%%%%%%%%%%%%%%%%%%%%%%%%%%%%%%%%%%%%%%%%%%%%%%%%%%%%
\frame[t]
{
  \frametitle{Parallel decomposition}
  	\framesubtitle{Heat diffusion problem}
      \begin{block}{The economy of decomposition}
    	\begin{figure}
   			\includegraphics[width=0.37\textwidth]{fig/ht-row.pdf}
   			\hfill
   			\includegraphics[width=0.37\textwidth]{fig/ht-cube.pdf}
      	\end{figure}
      	Partitioning as near-square rectangular blocks 
      	will give an advantage of $4*(\sqrt{p}$ - $1)N/\sqrt{p}$ times over 
      	partitioning as rows (where $p$ is perimeter and $N$ is number of partitions) 
      	in terms of the amount of data to be synchronized.     
  \end{block}
}
%%%%%%%%%%%%%%%%%%%%%%%%%%%%%%%%%%%%%%%%%%%%%%%%%%%%%%%%%%%%%%%%%%%%%%%%%%%%%%%%%%%%%%%%%%

%%%%%%%%%%%%%%%%%%%%%%%%%%%%%%%%%%%%%%%%%%%%%%%%%%%%%%%%%%%%%%%%%%%%%%%%%%%%%%%%%%%%%%%%%%
\frame[t]
{
  \frametitle{Parallel decomposition}
  	\framesubtitle{Heat diffusion problem}
      \begin{block}{Ghost cells}
      
      \begin{figure}
   		\includegraphics[width=0.475\textwidth]{fig/ht-grid-3.pdf}
   			\hfill
   		\includegraphics[width=0.475\textwidth]{fig/ht-grid-4.pdf}

      \end{figure} 
      Data visualization after sharing ghost cells among the processors. 		    
  \end{block}
}
%%%%%%%%%%%%%%%%%%%%%%%%%%%%%%%%%%%%%%%%%%%%%%%%%%%%%%%%%%%%%%%%%%%%%%%%%%%%%%%%%%%%%%%%%%


%%%%%%%%%%%%%%%%%%%%%%%%%%%%%%%%%%%%%%%%%%%%%%%%%%%%%%%%%%%%%%%%%%%%%%%%%%%%%%%%%%%%%%%%%%
\frame[t]
{
  \frametitle{Parallel decomposition}
  	\framesubtitle{Bioinformatics problem}
      \begin{block}{Case study: Bioinformatics}
   
  \end{block}
}
%%%%%%%%%%%%%%%%%%%%%%%%%%%%%%%%%%%%%%%%%%%%%%%%%%%%%%%%%%%%%%%%%%%%%%%%%%%%%%%%%%%%%%%%%%
%%%%%%%%%%%%%%%%%%%%%%%%%%%%%%%%%%%%%%%%%%%%%%%%%%%%%%%%%%%%%%%%%%%%%%%%%%%%%%%%%%%%%%%%%%
\frame[t]
{
  \frametitle{Parallel decomposition}
  	\framesubtitle{Bioinformatics problem}
      \begin{block}{Case study: Bioinformatics}
   
  \end{block}
	\note {This slide is intentionally left blank.} 
}
%%%%%%%%%%%%%%%%%%%%%%%%%%%%%%%%%%%%%%%%%%%%%%%%%%%%%%%%%%%%%%%%%%%%%%%%%%%%%%%%%%%%%%%%%%

\frame[t]
{
  \frametitle{Parallel decomposition}
  	\framesubtitle{Bioinformatics problem}
		\begin{block}{Biological data can be big, eg:}
		\begin{itemize}
		\item Gene expression data: DNA microarrays arrays now provide tens of thousands of values per sample
			\begin{figure}[!htb]
      			\includegraphics[width=230pt]{fig/cag_48_genome.jpg}	
			\end{figure}  
		\item Protein X-ray Crystalography
		\item DNA Sequence
		\end{itemize}
		\end{block}
}


\frame[t]
{
  \frametitle{Parallel decomposition}
  	\framesubtitle{Bioinformatics problem}
		\begin{block}{DNA Sequencing technology is outpacing Moore's Law}
			\begin{figure}[htb]
      			\includegraphics[width=0.9\textwidth]{fig/nbt.jpg}	
			\end{figure}  
		\end{block}
}


\frame[t]
{
  \frametitle{Parallel decomposition}
  	\framesubtitle{Bioinformatics problem}
		\begin{block}{Sequence Assembly}
      \begin{itemize}
      \item Raw sequence data is assembled into fewer, longer sequences by aligning overlaps. eg: 
         \begin{itemize}
            \item Each sequence read: 10s to 1000s of bases. 
            \item Longest human chromosome: 249 million bases.
         \end{itemize}
		\item It’s like a million piece jigsaw puzzle with duplicates: memory hungry and not fully parallelisable.
      \end{itemize}
		\end{block}
}

\frame[t]
{
  \frametitle{Parallel decomposition}
  	\framesubtitle{Bioinformatics problem}
		\begin{block}{Homology Search / Alignment}
		\begin{itemize}
		\item Q: What (parts of) known sequences are (parts of) your new sequence similar (i.e.: related) to?
	     	\item Another overlap alignment problem, but with fuzzier matching.
		\item Your new sequences vs known sequences, eg: the 37 million proteins in the ``refseq'' database.
		\item Can be parallelised by dividing up the queries or the reference sequences between the CPUs. 
      \item e.g: BLAST is multi-threaded, dividing up the database between the threads.
		\end{itemize}
		\end{block}
}

\frame[t]
{
  \frametitle{Parallel decomposition}
  	\framesubtitle{Bioinformatics problem}
		\begin{block}{Multiple Sequence Alignment}
      Aligning all the related sequences is also computationaly intensive even though the amount of data is generally not so large by this step.  
			\begin{figure}[!htb]
      			\includegraphics[width=0.9\textwidth]{fig/sequence-alignment.jpg}	
			\end{figure}  
		\end{block}
}

\frame[t]
{
  \frametitle{Parallel decomposition}
  	\framesubtitle{Bioinformatics problem}
		\begin{block}{Pylogenetic Reconstruction}
			Derive a tree of descent from multiple aligned sequences
			\begin{columns}[t]
			\column{0.45\textwidth}
			\begin{block}{Input}
            \begin{figure}[h]
      			\includegraphics[width=.7\textwidth]{fig/sequence-alignment.jpg}
            \end{figure}
			\end{block}
			\column{0.45\textwidth}
			\begin{block}{Output}
            \begin{figure}[h]
      			\includegraphics[width=.8\textwidth]{fig/phylogenetic-tree.png}	
            \end{figure}
			\end{block}  
			\end{columns}
		\end{block}
}

\frame[t]
{
  \frametitle{Parallel decomposition}
  	\framesubtitle{Bioinformatics problem}
		\begin{block}{Parallelise tree evaluation by:}
         \begin{itemize}
            \item Tree Branch - to some degree.
            \item Sequence Position - more generally.
         \end{itemize}
      \end{block}
		\begin{block}{Parallelise tree optimisation by:}
         \begin{itemize}
		      \item Parameter Space - eg: the rate of mutation may take on different values.
            \item Tree Space - the number of possible tree topologies increases quadratically.
         \end{itemize}
		\end{block}
}

%%%%%%%%%%%%%%%%%%%%%%%%%%%%%%%%%%%%%%%%%%%%%%%%%%%%%%%%%%%%%%%%%%%%%%%%%%%%%%%%%%%%%%%%%%
%%%%%%%%%%%%%%%%%%%%%%%%%%%%%%%%%%%%%%%%%%%%%%%%%%%%%%%%%%%%%%%%%%%%%%%%%%%%%%%%%%%%%%%%%%
\section{Classification}
%%%%%%%%%%%%%%%%%%%%%%%%%%%%%%%%%%%%%%%%%%%%%%%%%%%%%%%%%%%%%%%%%%%%%%%%%%%%%%%%%%%%%%%%%%
%%%%%%%%%%%%%%%%%%%%%%%%%%%%%%%%%%%%%%%%%%%%%%%%%%%%%%%%%%%%%%%%%%%%%%%%%%%%%%%%%%%%%%%%%%

%%%%%%%%%%%%%%%%%%%%%%%%%%%%%%%%%%%%%%%%%%%%%%%%%%%%%%%%%%%%%%%%%%%%%%%%%%%%%%%%%%%%%%%%%%
\frame[t]
{
  \frametitle{Parallel programming}
      \begin{block}{Memory Architecture}
      
		Architectural differences between memory architectures have implications 
       	on how we program.   
        \begin{itemize}
            \item Shared-memory systems uses a single address space allowing processors 
            to communicate through variables stored in a shared address space.    
            \item In distributed-memory system each processor has its own memory 
            module and connected over a high speed network for communication.
        \end{itemize}      
   
  \end{block}
}
%%%%%%%%%%%%%%%%%%%%%%%%%%%%%%%%%%%%%%%%%%%%%%%%%%%%%%%%%%%%%%%%%%%%%%%%%%%%%%%%%%%%%%%%%%

%%%%%%%%%%%%%%%%%%%%%%%%%%%%%%%%%%%%%%%%%%%%%%%%%%%%%%%%%%%%%%%%%%%%%%%%%%%%%%%%%%%%%%%%%%
%%%%%%%%%%%%%%%%%%%%%%%%%%%%%%%%%%%%%%%%%%%%%%%%%%%%%%%%%%%%%%%%%%%%%%%%%%%%%%%%%%%%%%%%%%
\subsection{Shared memory}
%%%%%%%%%%%%%%%%%%%%%%%%%%%%%%%%%%%%%%%%%%%%%%%%%%%%%%%%%%%%%%%%%%%%%%%%%%%%%%%%%%%%%%%%%%
%%%%%%%%%%%%%%%%%%%%%%%%%%%%%%%%%%%%%%%%%%%%%%%%%%%%%%%%%%%%%%%%%%%%%%%%%%%%%%%%%%%%%%%%%%

%%%%%%%%%%%%%%%%%%%%%%%%%%%%%%%%%%%%%%%%%%%%%%%%%%%%%%%%%%%%%%%%%%%%%%%%%%%%%%%%%%%%%%%%%%
\frame[t]
{
  \frametitle{Parallel programming}
      \begin{block}{Shared memory}

   \begin{itemize}%[<+-| alert@+>]
	\item Symmetric multiprocessing (SMP) : two or more identical processors are connected to a single shared main memory.
	\item This shared memory may be simultaneously accessed by single program using multiple threads.
	\item The most popular parallel programming paradigm is OpenMP.
   \end{itemize}
        \begin{center}
 			\includegraphics[width=100pt]{fig/shared-memory.png}
   		\end{center}  
   
  \end{block}
}
%%%%%%%%%%%%%%%%%%%%%%%%%%%%%%%%%%%%%%%%%%%%%%%%%%%%%%%%%%%%%%%%%%%%%%%%%%%%%%%%%%%%%%%%%%


%%%%%%%%%%%%%%%%%%%%%%%%%%%%%%%%%%%%%%%%%%%%%%%%%%%%%%%%%%%%%%%%%%%%%%%%%%%%%%%%%%%%%%%%%%
\frame[t]
{
  \frametitle{Parallel programming}
    \framesubtitle{Shared memory}
      \begin{block}{Shared memory: address system}
      
        \begin{center}
 			\includegraphics[width=200pt]{fig/shared-archi.png}
   		\end{center}  
   
  \end{block}
}
%%%%%%%%%%%%%%%%%%%%%%%%%%%%%%%%%%%%%%%%%%%%%%%%%%%%%%%%%%%%%%%%%%%%%%%%%%%%%%%%%%%%%%%%%%

%%%%%%%%%%%%%%%%%%%%%%%%%%%%%%%%%%%%%%%%%%%%%%%%%%%%%%%%%%%%%%%%%%%%%%%%%%%%%%%%%%%%%%%%%%
%%%%%%%%%%%%%%%%%%%%%%%%%%%%%%%%%%%%%%%%%%%%%%%%%%%%%%%%%%%%%%%%%%%%%%%%%%%%%%%%%%%%%%%%%%
\subsection{Distributed memory}
%%%%%%%%%%%%%%%%%%%%%%%%%%%%%%%%%%%%%%%%%%%%%%%%%%%%%%%%%%%%%%%%%%%%%%%%%%%%%%%%%%%%%%%%%%
%%%%%%%%%%%%%%%%%%%%%%%%%%%%%%%%%%%%%%%%%%%%%%%%%%%%%%%%%%%%%%%%%%%%%%%%%%%%%%%%%%%%%%%%%%

%%%%%%%%%%%%%%%%%%%%%%%%%%%%%%%%%%%%%%%%%%%%%%%%%%%%%%%%%%%%%%%%%%%%%%%%%%%%%%%%%%%%%%%%%%
\frame[t]
{
  \frametitle{Parallel programming}
      \begin{block}{Distributed memory}

   \begin{itemize}%[<+-| alert@+>]
	\item Multiple-processor computer system in which each process has its own private memory. 
	\item Computational tasks can only operate on local data.
	\item If remote data is required, the computational task must communicate with one or more remote processors.
	\item The most popular distribute memory programming paradigm is MPI.
   \end{itemize}
        \begin{center}
 			\includegraphics[width=150pt]{fig/dist-memory.png}
   		\end{center}     
  \end{block}
}
%%%%%%%%%%%%%%%%%%%%%%%%%%%%%%%%%%%%%%%%%%%%%%%%%%%%%%%%%%%%%%%%%%%%%%%%%%%%%%%%%%%%%%%%%%


%%%%%%%%%%%%%%%%%%%%%%%%%%%%%%%%%%%%%%%%%%%%%%%%%%%%%%%%%%%%%%%%%%%%%%%%%%%%%%%%%%%%%%%%%%
\frame[t]
{
  \frametitle{Parallel programming}
      \begin{block}{Distributed memory: address system}
        \begin{center}
 			\includegraphics[width=300pt]{fig/distri-archi.png}
   		\end{center}     
  \end{block}
}
%%%%%%%%%%%%%%%%%%%%%%%%%%%%%%%%%%%%%%%%%%%%%%%%%%%%%%%%%%%%%%%%%%%%%%%%%%%%%%%%%%%%%%%%%%


%%%%%%%%%%%%%%%%%%%%%%%%%%%%%%%%%%%%%%%%%%%%%%%%%%%%%%%%%%%%%%%%%%%%%%%%%%%%%%%%%%%%%%%%%%
\frame[t]
{
  \frametitle{Parallel programming}
      \begin{block}{OpenMP vs MPI}
        \begin{itemize}
            \item   OpenMP 
                \begin{itemize}
            		\item  Easy to parallelize existing codes without much coding effort.
        			\item Look for iterative operations
        			\item Use OpenMP directives (#pragma) to parallelize it. 
        			What it does is to create threads that will run parallel 
        			on different cores of the same node.
        		\end{itemize}
      
      		\item MPI
      		      \begin{itemize}
      		      	\item If you want to scale your application beyond the maximum 
      		      	number of cores available on a node. 
      
      				\item MPI is only a standard for message passing libraries based 
      				on the consensus of the MPI Forum, 
      				\item There are different implementations of it. 
      				\item Popular implementations are: OpenMPI, MPICH, Intel MPI.       
           		\end{itemize}
        \end{itemize}
  \end{block}
}
%%%%%%%%%%%%%%%%%%%%%%%%%%%%%%%%%%%%%%%%%%%%%%%%%%%%%%%%%%%%%%%%%%%%%%%%%%%%%%%%%%%%%%%%%%


%%%%%%%%%%%%%%%%%%%%%%%%%%%%%%%%%%%%%%%%%%%%%%%%%%%%%%%%%%%%%%%%%%%%%%%%%%%%%%%%%%%%%%%%%%
\frame[t]
{
  \frametitle{Parallel programming}
\begin{block}{Hybrid programming model}
    \begin{itemize}
      	\item Mix and match OpenMP and MPI
      	\item Use OpenMP directives within a single node and MPI communications over 
      	the network within the same program. 
    \end{itemize} 
  \end{block}
}
%%%%%%%%%%%%%%%%%%%%%%%%%%%%%%%%%%%%%%%%%%%%%%%%%%%%%%%%%%%%%%%%%%%%%%%%%%%%%%%%%%%%%%%%%%


%%%%%%%%%%%%%%%%%%%%%%%%%%%%%%%%%%%%%%%%%%%%%%%%%%%%%%%%%%%%%%%%%%%%%%%%%%%%%%%%%%%%%%%%%%
\frame[t]
{
  \frametitle{Acknowledgments}
        \begin{itemize}
            \item Slides developed by:
              \begin{itemize}
              	\item Jaison Mulerikkal, PhD
              	\item Peter Maxwell
				\end{itemize}
			\item For:
        \end{itemize}
  
\begin{center}
\vspace*{0.5cm}
 \includegraphics[width=210pt]{NeSI_img/nesi_logo.jpg}\\
 \includegraphics[width=50pt]{NeSI_img/logo-u-of-a.jpg}
 \includegraphics[width=50pt]{NeSI_img/logo-u-of-c.jpg}
 \includegraphics[width=50pt]{NeSI_img/logo-niwa.jpg}
 \includegraphics[width=50pt]{NeSI_img/logo-ministry-of-si.jpg} 
 \includegraphics[width=50pt]{NeSI_img/logo-manaaki-whenua.jpg}
 \includegraphics[width=50pt]{NeSI_img/logo-u-of-o.jpg}
\end{center}
}
%%%%%%%%%%%%%%%%%%%%%%%%%%%%%%%%%%%%%%%%%%%%%%%%%%%%%%%%%%%%%%%%%%%%%%%%%%%%%%%%%%%%%%%%%%
%%%%%%%%%%%%%%%%%%%%%%%%%%%%%%%%%%%%%%%%%%%%%%%%%%%%%%%%%%%%%%%%%%%%%%%%%%%%%%%%%%%%%%%%%%

%%%%%%%%%%%%%%%%%%%%%%%%%%%%%%%%%%%%%%%%%%%%%%%%%%%%%%%%%%%%%%%%%%%%%%%%%%%%%%%%%%%%%%%%%%
\frame[t]
{
  \frametitle{Acknowledgments}
      \begin{block}{Reference}
 		\begin{itemize}
   		 	\item  \href{https://computing.llnl.gov/tutorials/parallel_comp/}
   		 	{Blaise Barney, Introduction to Parallel Computing, LLNL}    
 		\end{itemize}
   
  \end{block}
}
%%%%%%%%%%%%%%%%%%%%%%%%%%%%%%%%%%%%%%%%%%%%%%%%%%%%%%%%%%%%%%%%%%%%%%%%%%%%%%%%%%%%%%%%%%



\end{document} 
